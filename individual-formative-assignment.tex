\documentclass{IEEEtran}

\usepackage[english]{babel}

\title{Philosophy and Ethics for Computing Scientists --- Individual Formative Assignment}
\author{Daan Brugmans S1080742}
\date{\today}

\begin{document}

\maketitle

\section*{What are the main takeaways from the course until now?}
Personally, I think that the two main takeaways from the course until now (that is, up and including lecture 3) are as follows:
\begin{enumerate}
    \item As a Computing Scientist, the topics of ethics and philosophy are very relevant to your field of study, because it is inevitable that you must eventually partake in topics, problems or dilemmas that are ethical or philosophical in nature, due to the decisions you must make during your research and/or development;
    \item By educating oneself on the topics of ethics and philosophy and the overlap between those fields and the field of Computing Science, a Computing Scientist can leverage ethical and philosophical frameworks and ways of thinking in their decision-making process when faced with topics, problems or dilemmas that are ethical or philosophical in nature in order to come to a decision that is both well-thought-out and substantiated by the use of applied ethical/philosophical frameworks and ways of thinking, which is to the benefit of all people affected by the decision in question.
\end{enumerate}

%TODO: Elaborate on my takeaways. What do they mean? Why do I think these are the main takeaways?

\end{document}
