\documentclass{IEEEtran}

\usepackage[english]{babel}

\title{Philosophy and Ethics for Computing Scientists --- Individual Formative Assignment}
\author{Daan Brugmans S1080742}
\date{\today}

\begin{document}

\maketitle

\section*{What are the main takeaways from the course until now?}
Personally, I think that the two main takeaways from the course until now (that is, up and including lecture 3) are as follows:
\begin{enumerate}
    \item As a Computing Scientist, the topics of ethics and philosophy are very relevant to your field of study, because it is inevitable that you must eventually partake in topics, problems or dilemmas that are ethical or philosophical in nature, due to the decisions you must make during your research and/or development;
    \item By educating oneself on the topics of ethics and philosophy and the overlap between those fields and the field of Computing Science, a Computing Scientist can leverage ethical and philosophical frameworks and ways of thinking in their decision-making process when faced with topics, problems or dilemmas that are ethical or philosophical in nature in order to come to a decision that is both well-thought-out and substantiated by the use of applied ethical/philosophical frameworks and ways of thinking, which is to the benefit of all people affected by the decision in question.
\end{enumerate}

In short, as a Computing Scientist, I have a responsibility to consider the ethical and philosophical consequences of my decisions prior to making them 
and by teaching myself basic frameworks and ways of thinking of ethics and philosophy, I can come to decisions that are well-tought-out from an ethical and/or philosophical perspective.

I think these are the main takeaways from the course, because they show to every student of Computing Science and Information Science that will eventually have to make decisions that are unavoidably ethical or philosophical in nature.
For example, I am currently following a course called Text and Multimedia Mining, a course about extracting knowledge from text (and multimedia) data.
In the very first lecture, we were taught to be ethically responsible in our research: 
we should never produce nor publish research on texts that, directly or indirectly, harm the people that wrote the texts.
I think that this is a good example of Computing Science ethics in practice.

Serious harm can be done to people when their written texts, their words, are presented to the public, especially when this is done so without proper care and consideration.
For example, for one of the assignments of the course, we had to analyze a collection of fanfiction data.
Fanfiction is fiction written by fans of existing media and is often an expression of their love and interest in the medium.
However, because of this, the contents of fanfiction can be very personal or inappropriate.
Because of this, it is important that these fanfiction texts are handled with care.
I would say that, in a context like this, authors should be anonymized to protect their identity when their works are used for purposes that they weren't meant for.
The stakes could be high for the author when a person dear to them discovers the work they have written and severely disapproves or disagrees with the contents of the author's work.
This should be considered when determining what data a researcher should use and how data should be preprocessed.

Though this might be a very specific example, I think that many, if not most, decisions that are ethical in nature in the field of Computing Science are that specific and local.
Many Computing Scientists are already aware of the big, unclear ethical dilemmas, like the Trolley Problem.
However, I think that they should also be aware of the small design decisions that have ethical consequences.
I think that the two main takeaways of this course help make students aware of these ethical topics, issues and dilemmas.

\section*{Are there things up to now that are unclear / not relevant according to you / etc.?}
No, I would not say so.

I think that the current course setup is appropriate for a short 3 EC course that should teach Computing Scientist students about the ethical and philosophical nature of their research and development.
That is, teaching the students about the most important ethical frameworks and ways of thinking, and how they can apply those techniques in their work.

%TODO: Elaborate on my takeaways. What do they mean? Why do I think these are the main takeaways?

\end{document}
